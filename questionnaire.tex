\documentclass{book}%
\usepackage[T1]{fontenc}%
\usepackage[utf8]{inputenc}%
\usepackage{lmodern}%
\usepackage{textcomp}%
\usepackage{lastpage}%
\input{structure}%
\usepackage{enumitem}%
%
%
%
\begin{document}%
\normalsize%
\newpage%
\thispagestyle{empty}%
\vskip-40mm	\includegraphics[scale=0.5]{logo.png} \\%
 \begin{flushright}  \vskip-20mm   Professeur khgk\vskip15mm  \end{flushright}%
Junia Maroc 2%
 \begin{flushright}  \vskip-7mm02/06/2022 \end{flushright}%
\begin{center}   \begin{Large}AZA\end{Large} \end{center}%
Durée : iui%
 \begin{center} { \large CONSIGNES SPÉCIFIQUES } \\ Lisez soigneusement les consignes ci-dessous afin de réussir au mieux cette épreuve : \end{center} %
\begin{itemize}%
\item%
L'usage de la calculatrice ou de tout autre appareil électronique est interdit.%
\item%
Aucun autre document que ce sujet et sa grille réponse n'est autorisé.%
\item%
Pour chacune des questions, indiquez sur la feuille de réponses ci-jointes, si les affirmations A, B, C et D sont (\textbf{V}) vraies ou (\textbf{F}) fausses en faisant une croix dans la colonne correspondant à votre choix. Vous ne pouvez pas faire de ratures. En cas d'erreur, utilisez la deuxième colonne de réponse. Si la deuxième colonne comporte au moins une réponse, la première colonne ne sera pas corrigée, c'est la deuxième qui sera prise en considération.%
\item%
Chaque réponse exacte est gratifiée de 3 points, tandis que chaque réponse fausse est pénalisée par -1 point. \\ 	Parmi les quatre propositions de chacune des questions \textbf{de 1 à 1}, une seule est vraie, les autres sont fausses. ( 3points par question) \\ 	Par exemple : Pour indiquer que l'affirmation $B$ est Vraie, cocher les cases comme suit:  \\ \begin{center}	\includegraphics[scale=0.8]{reponses.png} \end{center}%
\begin{exercise}%
\textbf{jbjhk }%
\begin{enumerate}[label=\textbf{\Alph*. }]%
\item%
jgvjh%
\item%
jhgjhj%
\item%
jhgjgv%
\item%
ghc%
\end{enumerate}%
\end{exercise}%
\end{itemize}%
\newpage%
\thispagestyle{empty}%
\begin{flushright}%
\begin{tabular}{|l|}%
\hline%
 \\%
Identifiant: $\qquad \qquad \qquad \qquad \qquad$ \\%
 \\%
\hline%
\end{tabular}%
\end{flushright}%
\begin{center}%
\begin{tabular}{| l l l l l |}%
\hline%
 & & & & \\%
Question 1\qquad \qquad\ & & & & \\%
 & A $\qquad \square \qquad$ & B $\qquad \square \qquad$ & C $\qquad \square \qquad$ & D $\qquad \square \qquad$ \\ %
 & & & &  \\%
\hline%
 & & & &  \\%
Choix 2 & A $\qquad \square \qquad$ & B $\qquad \square \qquad$ & C $\qquad \square \qquad$ & D $\qquad \square \qquad$ \\ %
 & & & &  \\%
\hline%
\end{tabular}%
\\ \vskip3mm%
\thispagestyle{empty}%
\end{center}%
\end{document}